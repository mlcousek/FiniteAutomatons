%%%  Ukázkový text a dokumentace stylu pro text závěrečné (bakalářské a
%%%  diplomové) práce na KI PřF UP v Olomouci
%%%  Copyright (C) 2012 Martin Rotter, <rotter.martinos@gmail.com>
%%%  Copyright (C) 2014 Jan Outrata, <jan.outrata@upol.cz>


%%  Pro získání PDF souboru dokumentu je třeba tento zdrojový text v
%%  LaTeXu přeložit (dvakrát) programem pdfLaTeX.

%%  V případě použití programu BibLaTeX pro tvorbu seznamu literatury
%%  je poté ještě třeba spustit program Biber s parametrem jméno
%%  souboru zdrojového textu bez přípony a následně opět (dvakrát)
%%  přeložit zdrojový text programem pdfLaTeX.

%%  Postup získání Postscriptového souboru je popsán v dokumentaci.


%%  Třída dokumentu implementující styl pro závěrečnou práci. Vybrané
%%  nepovinné parametry (ostatní v dokumentaci):

%%  'master' pro sazbu diplomové práce, jinak se sází bakalářská práce

%%  'program=kód' pro Váš studijní program/obor (specializaci), kódy
%%  pro diplomovou práci 'infoi' pro Informatiku (Obecná informatika),
%%  'infui' pro Informatiku (Umělá inteligence), 'ainfpst' pro
%%  Aplikovanou informatiku (Počítačové systémy a technologie), 'uinf'
%%  pro Učitelství informatiky pro střední školy, 'binf' pro
%%  Bioinformatiku, 'inf' pro Informatiku (bez specializací) a 'ainf'
%%  pro Aplikovanou informatiku (bez specializací), jinak je výchozí
%%  ainfvs pro Aplikovanou informatiku (Vývoj software), a pro
%%  bakalářskou práci 'infoi' pro Informatiku (Obecná informatika),
%%  'itp' pro Informační technologie v prezenční formě, 'itk' pro
%%  Informační technologie v kombinované formě, 'infv' pro Informatiku
%%  pro vzdělávání, 'binf' pro Bioinfomatiku, 'inf' pro Informatiku
%%  (bez specializací), 'ainfp' pro Aplikovanou informatiku (bez
%%  specializací) v prezenční formě, 'ainfk' pro Aplikovanou
%%  informatiku (bez specializací) v kombinované formě, jinak je
%%  výchozí infpvs pro Informatiku (Programování a vývoj software)

%%  'printversion' pro sazbu verze pro tisk (nebarevné logo a odkazy,
%%  odkazy s uvedením adresy za odkazem, ne odkazy do rejstříku),
%%  jinak verze pro prohlížeč

%%  'biblatex' pro zapnutí podpory pro sazbu bibliografie pomocí
%%  BibLaTeXu, jinak je výchozí sazba v prostředí thebibliography

%%  'language=jazyk' pro jazyk práce, jazyky english pro anglický,
%%  slovak pro slovenský, jinak je výchozí czech pro český

%%  'font=sans' pro bezpatkový font (Iwona Light), jinak je výchozí
%%  serif pro patkový (Latin Modern)

%%  'figures, tables, theorems a sourcecodes' pro sazbu seznamu
%%  obrázků, tabulek, vět a zdrojových kódů, jinak při =false se
%%  nesází (u theorems a sourcecodes výchozí)

\documentclass[
  master,
  program=ainfvs,
%  printversion,
  biblatex,
%  language=english,
%  font=sans,
  figures=false,
%  tables=false,
%  theorems,
%  sourcecodes,
  glossaries,
  index
]{kidiplom}

%% Informace pro úvodní strany. V jazyku práce (pokud není v komentáři
%% uvedeno česky) a anglicky. Uveďte všechny, u kterých není v
%% komentáři uvedeno, že jsou volitelné. Při neuvedení se použijí
%% výchozí texty. Text pro jiný než nastavený jazyk práce (nepovinným
%% parametrem language makra \documentclass, výchozí český) se zadává
%% použitím makra s uvedením jazyka jako nepovinného parametru.

%% Název práce, česky a anglicky. Měl by se vysázet na jeden řádek.
\title{Simulátor konečných automatů}
\title[english]{Finite automaton simulator}

%% Volitelný podnázev práce, česky a anglicky. Měl by se vysázet na
%% jeden řádek. Výchozí je prázdný.
%\subtitle{Ukázkový text a dokumentace stylu v \LaTeX{}u}
%\subtitle[english]{Sample text and documentation of the \LaTeX{} style}

%% Jméno autora práce. Makro nemá nepovinný parametr pro uvedení
%% jazyka.
\author{Bc. Jiří Mlčoušek}

%% Jméno vedoucího práce (včetně titulů). Makro nemá nepovinný
%% parametr pro uvedení jazyka.
\supervisor{doc. RNDr. Miroslav Kolařík, Ph.D.}

%% Volitelný rok odevzdání práce. Výchozí je aktuální (kalendářní)
%% rok. Makro nemá nepovinný parametr pro uvedení jazyka.
\yearofsubmit{2026}

%% Anotace práce, včetně anglické (obvykle překlad z jazyka
%% práce). Jeden odstavec!
\annotation{Tato diplomová práce se zabývá návrhem a implementací interaktivního simulátoru a vizualizátoru modelů z teorie formálních jazyků. Teoretická část práce podrobně mapuje Chomského hierarchii s důrazem na konečné automaty, zásobníkové automaty a Turingovy stroje. Práce dále popisuje klíčové algoritmy pro transformaci a minimalizaci těchto modelů, které tvoří jádro realizované aplikace určené pro podporu výuky teoretické informatiky.}

\annotation[english]{This master thesis focuses on the design and implementation of an interactive simulator and visualizer for models in formal language theory. The theoretical part provides a detailed map of the Chomsky hierarchy, emphasizing finite automata, pushdown automata, and Turing machines. Furthermore, the thesis describes key algorithms for transformation and minimization of these models, which form the core of the implemented application designed to support the education of theoretical computer science.}

%% Klíčová slova práce, včetně anglických. Oddělená (obvykle) středníkem.
\keywords{formální jazyky; konečné automaty; Turingovy stroje; vizualizace; simulace; algoritmy; Chomského hierarchie}
\keywords[english]{formal languages; finite automata; Turing machines; visualization; simulation; algorithms; Chomsky hierarchy}
%% Volitelná specifikace příloh textu práce, i anglicky. Výchozí je
%% 'elektronická data v systému katedry informatiky / electronic data
%% in system of department of computer science'.
%\supplements{nejlepší software všech dob}
%\supplements[english]{the best software of all times}

%% Volitelné poděkování. Stručné! Výchozí je prázdné. Makro nemá
%% nepovinný parametr pro uvedení jazyka.
\thanks{Rád bych poděkoval vedoucímu práce doc. RNDr. Miroslavu Kolaříkovi, Ph.D. za cenné rady, trpělivost a odborné vedení během zpracování této diplomové práce. Dále děkuji své rodině a blízkým za podporu při studiu.}
%% Cesta k souboru s bibliografií pro její sazbu pomocí BibLaTeXu
%% (zvolenou nepovinným parametrem biblatex makra
%% \documentclass). Použijte pouze při této sazbě, ne při (výchozí)
%% sazbě v prostředí thebibliography.
\bibliography{bibliografie.bib}

%% Další dodatečné styly (balíky) potřebné pro sazbu vlastního textu
%% práce.
\usepackage{lipsum}
\usepackage{longtable}

\begin{document}
%% Sazba úvodních stran -- titulní, s bibliografickými údaji, s
%% anotací a klíčovými slovy, s poděkováním a prohlášením, s obsahem a
%% se seznamy obrázků, tabulek, vět a zdrojových kódů (pokud jejich
%% sazba není vypnutá).
\maketitle

%% Vlastní text závěrečné práce. Pro povinné závěry, před přílohami,
%% použijte prostředí kiconclusions. Povinná je i příloha s obsahem
%% elektronických dat.

%% -------------------------------------------------------------------

\section{Úvod}
Teorie formálních jazyků a automatů tvoří teoretický fundament moderní informatiky. Porozumění těmto konceptům je nezbytné pro návrh překladačů, analýzu složitosti algoritmů i pro pochopení limitů samotné vyčíslitelnosti. Přestože je tato oblast primárně matematická, její praktická aplikace v softwarovém inženýrství je nepopiratelná.

Motivací pro tuto diplomovou práci je vytvořit moderní, interaktivní nástroj, který by studentům a zájemcům o teoretickou informatiku umožnil vizualizovat abstraktní modely, jako jsou konečné automaty či Turingovy stroje. Tato práce si klade za cíl nejen popsat teoretické základy, ale také implementovat algoritmy, které tyto modely transformují, minimalizují a simulují v reálném čase.

%% --- TEORETICKÁ ČÁST ---
\section{Teoretický základ}
Tato kapitola shrnuje matematický aparát nezbytný pro pochopení problematiky formálních jazyků. Text čerpá primárně zavedenou terminologii z publikací \cite{cerna2002} a \cite{jancar2007}.

\subsection{Formální jazyky a Chomského hierarchie}
Základem teorie je definice abecedy $\Sigma$ jako konečné množiny symbolů. Slovo je konečná posloupnost těchto symbolů a jazyk je libovolná množina slov.

Klíčovým konceptem je Chomského hierarchie, která dělí jazyky do čtyř tříd podle jejich generativní síly:
\begin{enumerate}
    \item \textbf{Typ 0 (Rekurzivně spočetné):} Jazyky přijímané Turingovými stroji.
    \item \textbf{Typ 1 (Kontextové):} Jazyky přijímané lineárně omezenými automaty.
    \item \textbf{Typ 2 (Bezkontextové):} Jazyky popsané bezkontextovými gramatikami a přijímané zásobníkovými automaty.
    \item \textbf{Typ 3 (Regulární):} Nejslabší třída, přijímaná konečnými automaty.
\end{enumerate}

\subsection{Konečné automaty (DFA a NFA)}
Konečný automat je nejjednodušším modelem s konečnou pamětí. Formálně definujeme deterministický konečný automat (DFA) jako pětici $M = (Q, \Sigma, \delta, q_0, F)$.

\subsection{Determinizace a minimalizace}
Nedeterministické automaty (NFA) umožňují více přechodů pro jeden symbol. Pro účely simulace je však často nutná jejich determinizace pomocí tzv. \textit{podmnožinové konstrukce}. Po determinizaci je dalším krokem minimalizace automatu, která vede na kanonický tvar s nejmenším možným počtem stavů při zachování přijímaného jazyka \cite{balun2025}.

\subsection{Regulární výrazy a jejich převod}
Regulární výrazy poskytují syntakticky úsporný způsob popisu regulárních jazyků. V implementované aplikaci je využíván \textbf{Thompsonův algoritmus} pro převod regulárního výrazu na NFA s $\varepsilon$-kroky, což umožňuje vizualizaci postupu od textového zápisu ke grafové reprezentaci.

\subsection{Bezkontextové jazyky a zásobník}
Pro jazyky typu 2 (např. vyvážené závorky) již konečný automat nestačí. Je nutné zavést zásobníkový automat (PDA), který rozšiřuje konečný automat o paměť typu LIFO. V práci se zaměříme na simulaci práce se zásobníkovou abecedou $\Gamma$.

\subsection{Turingovy stroje}
Turingův stroj (TM) představuje univerzální model výpočtu. Skládá se z řídicí jednotky, nekonečné pásky a čtecí hlavy. V simulátoru je kladen důraz na vizualizaci konfigurace stroje, která je definována jako trojice (stav, obsah pásky, pozice hlavy) \cite{jancar2007}.


\section{Návrh systému}
- Požadavky na systém.
- Uživatelské rozhraní a jeho funkce.
- Grafická reprezentace automatů.
- Struktura dat pro uchování automatů.

\section{Implementace}
- Použité technologie.
- Realizace interaktivního editoru automatů.
- Implementace krokového režimu.
- Ukládání a načítání automatů.
- Minimalizace automatů.
- Generování automatů z regulárních výrazů.

\section{Testování a experimenty}
- Testovací scénáře.
- Výsledky testování.
- Srovnání s jinými existujícími nástroji.





%% Závěry práce. V jazyce práce a anglicky. Text pro jiný než
%% nastavený jazyk práce (nepovinným parametrem language makra
%% \documentclass, výchozí český) se zadává použitím makra s uvedením
%% jazyka jako nepovinného parametru.
\begin{kiconclusions}
Závěr práce v \uv{českém} jazyce.
\end{kiconclusions}

\begin{kiconclusions}[english]
Thesis conclusions in \uv{English}.
\end{kiconclusions}

%% Přílohy obsahu textu práce, za makrem \appendix.
\appendix

\section{První příloha}
Text první přílohy

\section{Druhá příloha}
Text druhé přílohy

%% Obsah elektronických dat. Poslední příloha. Upravte podle vlastní
%% práce!
\section{Obsah elektronických dat} \label{sec:ObsahData}

Na samotném konci textu práce je uveden stručný popis obsahu
elektronických dat odevzdaných v systému katedry informatiky spolu s
textem. Tato data jsou nedílnou součástí práce a tvoří (datovou)
přílohu textu práce. Povinné položky struktury dat jsou:

\begin{description}

\item[\texttt{text/}] \hfill \\
  Adresář s textem práce ve formátu PDF, vytvořený s~použitím
  závazného stylu KI PřF UP v~Olomouci pro závěrečné práce, včetně
  všech (textových) příloh, a~všechny soubory potřebné pro
  bezproblémové vytvoření PDF dokumentu textu (případně v~ZIP
  archivu), tj.~zdrojový text textu a příloh, vložené obrázky, apod.

\item[\texttt{README.*}] \hfill \\
  Textový soubor (s příponou např. \texttt{.txt}) s informacemi o
  opakovatelném způsobu použití ostatních dat práce -- typicky plně
  reprodukovatelný co nejúplnější funkční postup zprovoznění software
  vytvořeného v~rámci práce, tzn. jeho případné instalace/nasazení a
  spuštění, včetně uvedení všech požadavků pro bezproblémový provoz;
  za zprovoznění software se nepovažuje zpřístupnění (např. po
  Internetu) již někde zprovozněného software.

\item[\texttt{*}] \hfill \\
  Adresáře a soubory s veškerými ostatními autorskými daty práce
  (případně v~ZIP archivu) -- typicky spustitelné a další soubory
  software vytvořeného v rámci práce potřebné pro bezproblémový provoz
  software, případně jeho instalační program, a kompletní zdrojové
  texty software a další data nutná pro plně reprodukovatelné korektní
  vytvoření spustitelných souborů.

\end{description}

Dále mohou data obsahovat například:

\begin{itemize}

%\item[\texttt{data/}] \hfill \\
\item
  ukázková a~testovací data použitá v~práci nebo pro potřeby posouzení
  práce v rámci její obhajoby,

%\item[\texttt{literature/}] \hfill \\
\item
  položky bibliografie v elektronické podobě, příp.~jiná relevantní literatura
  a dokumentace vztahující se k~práci,

%\item[\texttt{install/}] \hfill \\
\item
  cizí data (software) potřebná pro bezproblémové použití autorských
  dat práce (software), která nejsou standardní součástí
  předpokládaného (softwarového) vybavení uživatele.

\end{itemize}

U~veškerých cizích obsažených materiálů jejich
zahrnutí dovolují podmínky pro jejich veřejné šíření nebo přiložený souhlas
držitele práv k užití. Pro všechny použité (a~citované) materiály,
u~kterých toto není splněno a~nejsou tak obsaženy, je uveden
jejich zdroj, např.~webová adresa, v~bibliografii nebo textu práce
nebo souboru \texttt{README.*}.

%% -------------------------------------------------------------------

%% Sazba volitelného seznamu zkratek, za přílohami.
\printglossary

%% Sazba povinné bibliografie, za přílohami (případně i za seznamem
%% zkratek). Při použití BibLaTeXu použijte makro
%% \printbibliography. jinak prostředí thebibliography. Ne obojí!

%% Sazba i v textu necitovaných zdrojů, při použití
%% BibLaTeXu. Volitelné.
\nocite{*}
%% Vlastní sazba bibliografie při použití BibLaTeXu.
\printbibliography

%% Bibliografie, včetně sazby, při NEpoužití BibLaTeXu.
% \begin{thebibliography}{9}
%\bibitem{kniha2} \uppercase{Hawke}, Paul. NanoHttpd: Light-weight HTTP server designed for embedding in other applications. GitHub [online]. 2014-05-12. [cit. 2014-12-06]. Dostupné z: \url{https://github.com/NanoHttpd/nanohttpd}
%
%\bibitem{jeske13} \uppercase{Jeske}, David; \uppercase{Novák}, Josef. Simple HTTP Server in \csharp: Threaded synchronous HTTP Server abstract class, to respond to HTTP requests. CodeProject: For those who code [online]. 2014-05-24. [cit. 2014-12-06]. Dostupné z: \url{http://www.codeproject.com/Articles/137979/Simple-HTTP-Server-in-C}
%
%\bibitem{uzis2012} \uppercase{ÚSTAV ZDRAVOTNICKÝCH INFORMACÍ A STATISTIKY ČR}. Lékaři, zubní lékaři a farmaceuti 2012 [online]. Praha 2, Palackého náměstí 4: Ústav zdravotnických informací a statistiky ČR, 2012 [cit. 2014-12-06]. ISBN 978-80-7472-089-5. Dostupné z: \url{http://www.uzis.cz/publikace/lekari-zubni-lekari-farmaceuti-2012}
% \end{thebibliography}

%% Sazba volitelného rejstříku, za bibliografií.
\printindex

\end{document}

%%% Local Variables:
%%% mode: latex
%%% TeX-master: t
%%% End:
